\section{Teaching Experience}

% --- Bioinformatics and Computational Biology Education ---
\subsection{Bioinformatics and Computational Biology Education}

\cventry{2020--Present}{Course Director}{CDRL Bioinformatics Bootcamp}{}{}{
    Led the development and instruction of a hybrid bootcamp designed for undergraduates, graduate students, medical students, residents, and physicians interested in computational biology. Delivered live lectures and designed asynchronous learning tracks.\\
    \textbf{Learning Goals:} Introduce R programming, data manipulation, and core bioinformatics workflows using Bioconductor. Participants completed reproducible projects with live support.
}

\cventry{2022--2023}{Instructor of Record (Computational Component)}{AI in Medicine (M3 Elective)}{University of Toledo College of Medicine}{}{
    Designed and taught the computational curriculum for M3 medical students, covering R programming, basic statistical methods, and introductory machine learning techniques.\\
    Based on \textit{An Introduction to Statistical Learning}, with assessments submitted and autograded via GitHub.
}

\cventry{2023--2025}{Lecturer and Research Mentor}{Summer Biomedical Science Program}{}{CDRL}{
    Delivered 12 (2023), 11 (2024), and 5 (2025) lectures to high school students in a summer research program. Mentored project teams each year; all teams successfully published a research paper.\\
    \textbf{Lecture Topics:} R and RStudio, differential gene expression, GSEA and pathway analysis, transcriptomics, reproducible reporting.
}

\cventry{2020}{Course Designer and Instructor}{CDRL Spring Webinar Series}{}{CDRL}{
    Developed and delivered a webinar series in response to COVID-19 research disruptions. Aimed at graduate students and scientists.\\
    \textbf{Goal:} Empower researchers to continue analytical work without lifting a pipette.\\
    Delivered 3 of the 8 sessions. Lectures were recorded and archived.
}

% --- Scientific Workshops ---
\subsection{Scientific Workshops}

\cventry{2019}{Workshop Instructor}{SPSS for Medical Research}{}{University of Health Sciences}{
    Delivered hands-on training in SPSS to fellow medical students, focused on statistical analysis and research project planning.
}

\cventry{2020}{Workshop Instructor}{Data Science: A Crossroad between Research and Data Analysis}{}{CDRL}{
    Three-day workshop for graduate students introducing R, Bioconductor, and reproducible workflows.\\
    \textbf{Outcome:} Participants produced an R Markdown analysis of a synthetic dataset.
}

\cventry{2021}{Instructor}{Single-Cell RNA-seq Analysis Workshop}{}{CDRL}{
    Taught single-cell transcriptomic analysis methods including clustering, normalization, and visualization using Seurat. Aimed at early-career researchers.
}

% --- Guest Lectures and Public Engagement ---
\subsection{Guest Lectures and Public Engagement}

\cventry{2021}{Guest Lecturer}{Rationality: How to Change Your Mind}{}{Irrtiqa}{
    Public lecture on epistemology, heuristics, and belief revision. Delivered live on YouTube and archived online.
}

\cventry{2022}{Guest Lecturer}{How to Change Your Thinking}{}{Lincoln Corner Multan / Pak-US Alumni Network}{
    Invited lecture on critical thinking skills, decision-making frameworks, and cognitive biases for youth and community leaders.
}

\cventry{2021}{Guest Lecturer}{Cybersecurity for Civil Society Organizations}{}{Alliance of Inclusive Muslims}{
    Capacity-building session on digital security and data protection practices for NGOs and advocacy groups.
}

\cventry{2020}{Guest Lecturer}{Thalassemia Awareness: Prevention and Management}{}{Rotaract Club of BZU}{
    Delivered a talk to undergraduate students about genetic risk factors, screening, and treatment of thalassemia.
}

\cventry{2022}{Guest Lecturer}{Psychiatric Genomics and Bioinformatics}{}{Department of Psychiatry, University of Toledo}{
    Presented an introduction to psychiatric transcriptomics, data resources, and statistical approaches in genomic studies.
}

% --- Youth and Leadership Training ---
\subsection{Youth and Leadership Training}

\cventry{2011--2012}{Founder and Director}{Youth Leadership Camp}{}{Irrtiqa}{
    Designed and led a 12-week leadership training program for high school and undergraduate students.\\
    Curriculum covered ethics, communication, critical thinking, and civic engagement. Held over two summers with weekly interactive sessions.
}
