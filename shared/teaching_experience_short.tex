\section{Teaching Experience}

\cventry{2020--Present}{Course Director}{CDRL Bioinformatics Bootcamp}{}{}{
    Led the development and instruction of a hybrid bootcamp designed for undergraduates, graduate students, medical students, residents, and physicians interested in computational biology. Delivered live lectures and designed asynchronous learning tracks.\\
    \textbf{Learning Goals:} Introduce R programming, data manipulation, and core bioinformatics workflows using Bioconductor. Participants completed reproducible projects with live support.
}

\cventry{2022--2023}{Instructor of Record (Computational Component)}{AI in Medicine (M3 Elective)}{University of Toledo College of Medicine}{}{
    Designed and taught the computational curriculum for M3 medical students, covering R programming, basic statistical methods, and introductory machine learning techniques.\\
    Based on \textit{An Introduction to Statistical Learning}, with assessments submitted and autograded via GitHub.
}

\cventry{2023--2025}{Lecturer and Research Mentor}{Summer Biomedical Science Program}{}{CDRL}{
    Delivered 12 (2023), 11 (2024), and 5 (2025) lectures to high school students in a summer research program. Mentored project teams each year; all teams successfully published a research paper.\\
    \textbf{Lecture Topics:} R and RStudio, differential gene expression, GSEA and pathway analysis, transcriptomics, reproducible reporting.
}

\cventry{2020}{Course Designer and Instructor}{CDRL Spring Webinar Series}{}{CDRL}{
    Developed and delivered a webinar series in response to COVID-19 research disruptions. Aimed at graduate students and scientists.\\
    \textbf{Goal:} Empower researchers to continue analytical work without lifting a pipette.\\
    Delivered 3 of the 8 sessions. Lectures were recorded and archived.
}
